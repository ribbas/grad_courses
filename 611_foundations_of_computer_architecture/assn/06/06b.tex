\documentclass[12pt]{article}
\usepackage[margin=1.0in]{geometry}
\usepackage{setspace}
\usepackage{titling}

\usepackage{amsmath,amsthm,amssymb}
\usepackage{graphicx}
\usepackage{enumerate}
\usepackage[shortlabels]{enumitem}


\setstretch{1.5}
\setlength{\droptitle}{-8em}

\usepackage{listings}
\usepackage{/home/sabbir/texmf/mips}
% the following is needed for syntax highlighting
\usepackage{color}

\definecolor{dkgreen}{rgb}{0,0.6,0}
\definecolor{gray}{rgb}{0.5,0.5,0.5}
\definecolor{mauve}{rgb}{0.58,0,0.82}

\lstset{ %
  language=[mips]Assembler,       % the language of the code
  % basicstyle=\footnotesize,       % the size of the fonts that are used for the code
  numbers=left,                   % where to put the line-numbers
  numberstyle=\tiny\color{gray},  % the style that is used for the line-numbers
  showspaces=false,               % show spaces adding particular underscores
  showstringspaces=false,         % underline spaces within strings
  showtabs=false,                 % show tabs within strings adding particular underscores
  tabsize=4,                      % sets default tabsize to 2 spaces
  captionpos=b,                   % sets the caption-position to bottom
  breaklines=true,                % sets automatic line breaking
  breakatwhitespace=false,        % sets if automatic breaks should only happen at whitespace
  title=\lstname,                 % show the filename of files included with \lstinputlisting;
belowskip=-0.8 \baselineskip,                                  % also try caption instead of title
  keywordstyle=\color{blue},          % keyword style
  commentstyle=\color{dkgreen},       % comment style
  stringstyle=\color{mauve},         % string literal style
  escapeinside={\%*}{*)},            % if you want to add a comment within your code
  morekeywords={*,...}               % if you want to add more keywords to the set
}

\begin{document}

  \title{605.611 - Foundations of Computer Architecture \\ Assignment 06 - Branching and Jumps\vspace{-0.5em}}
  \author{Sabbir Ahmed}
  \maketitle
  \vspace{-1em}

  \begin{enumerate}

    \item Translate statements 1, 2, and 5 into machine code.  Assumes the program loads at 0x0040 \ 0000.

    \begin{lstlisting}
label_1:
    beq $zero, $zero, label_2
    j label_3
    addi $t0, $t1, 100
    sub $t0, $t1, $t2
label_2:
    beq $zero, $zero, label_1
label_3:
    addi $t0, $t1, 100
    \end{lstlisting}

    \textbf{Answer:} The statements 1, 2, and 5 are

    \begin{lstlisting}
beq $zero, $zero, label_2
j label_3
beq $zero, $zero, label_1
    \end{lstlisting}

    Putting the program with its address space:

    \begin{center}
      \begin{tabular}{ |l|l| } 
        \hline
        \textbf{Address} & \textbf{Instruction} \\
        \hline
        0x0040 \ 0000 & \begin{tabular}{@{}l@{}}
          label\_1:\\ beq \$zero, \$zero, label\_2
        \end{tabular} \\
        \hline
        0x0040 \ 0004 & j label\_3 \\
        \hline
        0x0040 \ 0008 & addi \$t0, \$t1, 100 \\
        \hline
        0x0040 \ 000C & sub \$t0, \$t1, \$t2 \\
        \hline
        0x0040 \ 0010 & \begin{tabular}{@{}l@{}}
          label\_2:\\ beq \$zero, \$zero, label\_1
          \end{tabular} \\
        \hline
        0x0040 \ 0014 & \begin{tabular}{@{}l@{}}
          label\_3:\\ addi \$t0, \$t1, 100
          \end{tabular} \\
        \hline
      \end{tabular}
    \end{center}
    % \vspace{0.5em}

    The addresses of the labels are as follows:

    \begin{center}
      \begin{tabular}{ |l|l| } 
        \hline
        \textbf{Address} & \textbf{Instruction} \\
        \hline
          0x0040 \ 0000 & label\_1 \\
          0x0040 \ 0010 & label\_2 \\
          0x0040 \ 0014 & label\_3 \\
        \hline
      \end{tabular}
    \end{center}
    % \vspace{0.5em}

    Replacing addresses in absolute jumps:

    \begin{center}
      \begin{tabular}{ |l|l| } 
        \hline
        \textbf{Address} & \textbf{Instruction} \\
        \hline
        0x0040 \ 0000 & beq \$zero, \$zero, label\_2 \\
        \hline
        0x0040 \ 0004 & j 0x0040 \ 0014 \\
        \hline
        0x0040 \ 0010 & beq \$zero, \$zero, label\_1\\
        \hline
      \end{tabular}
    \end{center}

    For the relative branches, label\_2 is +3 PC (0x0000 \ 0003) from label\_1 and label\_1 is -5 PC (0xFFFF \ FFFB) from label\_2.  Therefore,

    \begin{center}
      \begin{tabular}{ |l|l| } 
        \hline
        \textbf{Address} & \textbf{Instruction} \\
        \hline
        0x0040 \ 0000 & beq \$zero, \$zero, 0x0000 \ 0003 \\
        \hline
        0x0040 \ 0004 & j 0x0040 \ 0014 \\
        \hline
        0x0040 \ 0010 & beq \$zero, \$zero, 0xFFFF FFFB \\
        \hline
      \end{tabular}
    \end{center}

    \begin{enumerate}

      \item \texttt{beq \$zero, \$zero, 0x0000 \ 0003}
      \begin{align*}
        \text{opcode(beq)} &= 04_{16} \\
        &= 000100_2\\
        \text{rs(\$zero)} &= 00000_2 \\
        \text{rt(\$zero)} &= 00000_2 \\
        \text{sign\_ext\_imm(3)} &\Rightarrow 0003_{16}\\
        &\Rightarrow 000100 \ 00000 \ 00000_2 \ | \ 0003_{16}\\
        &\Rightarrow 0001 \ 0000 \ 0000 \ 0000 \ | \ 0003_{16}\\
        &\Rightarrow 10000003_{16}
      \end{align*}

      \item \texttt{j 0x0040 \ 0014}
      \begin{align*}
        \text{opcode(j)} &= 02_{16} \\
        &= 000010_2\\
        \text{imm(0x0040 \ 0014)} &\Rightarrow 0040 \ 0014_{16}\\
        &\Rightarrow 000010 \ 0000_2 \ | \ 40 \ 0014_{16} \ | \ 00_2 \\
        &\Rightarrow 000010 \ 0000 \ 0100  \ 0000 \ 0000 \ 0000 \ 0001 \ 0100 \ 00 \\
        &\Rightarrow 0000 \ 1000 \ 0001 \ 0000 \ 0000 \ 0000 \ 0000 \ 0101 \ 0000\\
        &\Rightarrow 08100005_{16}
      \end{align*}

      \setcounter{enumi}{5}
      \item \texttt{beq \$zero, \$zero, 0xFFFF \ FFFB}
      \begin{align*}
        \text{opcode(beq)} &= 04_{16} \\
        &= 000100_2\\
        \text{rs(\$zero)} &= 00000_2 \\
        \text{rt(\$zero)} &= 00000_2 \\
        \text{sign\_ext\_imm(-5)} &\Rightarrow FFFB_{16}\\
        &\Rightarrow 000100 \ 00000 \ 00000_2 \ | \ FFFB_{16}\\
        &\Rightarrow 0001 \ 0000 \ 0000 \ 0000 \ | \ FFFB_{16}\\
        &\Rightarrow 1000FFFB_{16}
      \end{align*}

    \end{enumerate}

    \item Explain, in your own words, how a 26 bit address in the \texttt{jmp} instruction can access any executable statement in a program. This must be your own words. Copying something from the internet does not count.

    \textbf{Answer:} All of the instructions in a MIPS machine allocate 6 bits for their opcodes. This limits 26 bits for the address for the \texttt{jmp} instruction. However, the memory is configured in a MIPS machine such that some areas are protected and reserved from the program text, i.e. the kernel space, operating system instructions, etc. The program text is limited to 0x1000 \ 0000, which leaves a highest address of 0x0fff \ ffff. The most significant bit of the addresses are therefore always 0 (in hexadecimal), and 4-bits are regained. The other 2 bits come for the 2 zeros on the least significant bits that account for the word alignment of the addresses.

  \end{enumerate}

\end{document}
