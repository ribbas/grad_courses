\documentclass[11pt]{article}

\usepackage{graphicx}
\usepackage[fleqn]{amsmath}
\usepackage{listings}
\usepackage{array}
\usepackage[autostyle, english = american]{csquotes}
\MakeOuterQuote{"}

% Margins
\topmargin=-0.45in
\evensidemargin=0in
\oddsidemargin=0in
\textwidth=6.5in
\textheight=9.0in
\headsep=0.25in
\setlength{\mathindent}{0pt}

\newcounter{divline}
\def\rlwd{.5pt} \def\rlht{\dimexpr\dp\strutbox+\ht\strutbox} \def\rldp{.75ex}
\newcommand\bindiv[3][\relax]{%
  \ifx\relax#1\stepcounter{divline}\else\setcounter{divline}{#1}\fi%
  \mbox{}\hspace{\thedivline\dimexpr1ex}#2~\setbox0=\hbox{~$#3$}%
  \dumbstackengine{-\rlwd}{\rule[-\rldp]{\rlwd}{\rlht}~#3}{\rule{\dimexpr4pt+\wd0}{\rlwd}}%
}
\def\remainder#1{\stepcounter{divline}%
  \mbox{}\hspace{\dimexpr1ex+\thedivline\dimexpr1ex}~#1\setcounter{divline}{0}}
\makeatletter
\global\newlength\@stackedboxwidth
\newlength\@boxshift
\newsavebox\@addedbox
\newsavebox\@anchorbox
\newcommand*\dumbstackengine[3]{%
    \sbox{\@anchorbox}{$#2$}%
    \sbox{\@addedbox}{$#3$}%
    \setlength{\@stackedboxwidth}{\wd\@anchorbox}%
      \ifdim\wd\@addedbox>\@stackedboxwidth%
        \setlength{\@stackedboxwidth}{\wd\@addedbox}%
      \fi%
        \setlength{\@boxshift}{\dimexpr-\dp\@anchorbox -\ht\@addedbox -#1}%
        \usebox{\@anchorbox}%
        \hspace{-\wd\@anchorbox}%
        \raisebox{\@boxshift}{\usebox{\@addedbox}}%
        \hspace{-\wd\@addedbox}%
        \hspace{\@stackedboxwidth}%
}


\title{605.744: Information Retrieval \\ Problem Set (Module 3)}
\author{Sabbir Ahmed}
\date{\today}

\begin{document}
\maketitle

    \begin{enumerate}

        \item (20\%) Examine Table 5.1 in the text. The table reports the effect of various tokenization choices on the size of the index built from the Reuters RCV1 corpus.

        \begin{enumerate}
            \item Expressed as a percentage of the original vocabulary size, how big is the vocabulary size (i.e., the number of indexing terms) after removing numbers, case folding, stopword removal, and then stemming with the Porter stemmer?

            \textbf{Answer:} The original vocabulary size was 484,494. The size after being reduced through the different levels of preprocessing is 322,383, which is 66.54\% of the original vocabulary size.

            \item After removing numbers and case normalization, there are 96.97 million posting list entries in the inverted file. Removing 150 stop words reduces this to 67.00 million. If stemming is then performed, the number of posting lists entries reduces to 63.81 million. Explain why stemming makes a difference in the number of entries. A short example may be helpful.

            \textbf{Answer:} Stemming terms leads to loss of information regardless of context. When terms are stemmed, depending on the algorithm used, the term may have its suffixes replaced with a simple fragment or have it entirely removed. Different words with similar prefixes of a term usually stem to identical terms, for example, both of the words "intern" and "internal" get stemmed to "intern" using a Porter stemmer. Although those words are unrelated, they get stemmed to the common term and the vocabulary size gets reduced by 1. Different conjugations of words also get reduced to their common stemmed word, such as "transparent" and "transparency" get stemmed to "transpar" using a Porter stemmer.

            \item Is the mean posting list length longer or shorter after stemming is performed? Offer a brief explanation for the difference.

            \textbf{Answer:} The mean posting list length stays the same after stemming is performed. Once stemming is performed, some words get normalized to their common roots. During the replacement, a token that was once pointing to one Doc-ID now gets referenced by all of the Doc-IDs that had their tokens stemmed to the common root. Using the same example from the previous answer, if the words "intern" and "internal" have existed in separated documents, their posting lists would only contain their corresponding Doc-ID. After being stemmed to "intern", its posting list now references that Doc-IDs of the posting list of "internal". However, the total number of Doc-IDs is preserved.

        \end{enumerate}

        \item (30\%) Express the numbers {32, 57, and 800} two ways: using a 12-bit binary representation and using gamma coding. You must follow the method for computing gamma described in the text and presented in the lecture materials.
        
        I strongly recommend learning to do this by hand, but you may write (and provide) a short computer program if you prefer - but do not use a program that you did not write yourself.

        \textbf{Answer:}
        Gamma coding formula: $unary(floor(log_2(x))), binary(x-2^{(floor(log_2(x)))})$

        \begin{enumerate}
            \item 32

            \textbf{Answer:}

            \begin{itemize}
                \item Converting to binary:

                \begin{tabular}{lcc}
                    \bindiv{2}{32} && Remainders\\
                    \bindiv{2}{16} &---& 0\\
                    \bindiv{2}{8} &---& 0\\
                    \bindiv{2}{4} &---& 0\\
                    \bindiv{2}{2} &---& 0\\
                    \bindiv{2}{0} &---& 0\\
                    \remainder{1} &---& 1 
                \end{tabular}

                12-bit representation: $0000 \ 0010 \ 0000$

                \item Converting to gamma coding:
                \begin{align*}
                    unary(floor(log_2(32))) &= unary(5) \\
                    &= 111110 \\
                    binary(x-2^{(floor(log_2(x)))}) &= binary(32 - 32) \\
                    &= 00000 \\
                    gamma(32) &= 0111 \ 1100 \ 0000
                \end{align*}
            \end{itemize}

            \item 57

            \textbf{Answer:}

            \begin{itemize}
                \item Converting to binary:

                \begin{tabular}{lcc}
                    \bindiv{2}{57} && Remainders\\
                    \bindiv{2}{28} &---& 1\\
                    \bindiv{2}{14} &---& 0\\
                    \bindiv{2}{7} &---& 0\\
                    \bindiv{2}{3} &---& 1\\
                    \bindiv{2}{1} &---& 1\\
                    \remainder{1} &---& 1
                \end{tabular}

                12-bit representation: $0000 \ 0011 \ 1001$

                \item Converting to gamma coding:
                \begin{align*}
                    unary(floor(log_2(57))) &= unary(5) \\
                    &= 111110 \\
                    binary(x-2^{(floor(log_2(x)))}) &= binary(57 - 32) \\
                    &= binary(25) \\
                    &= 11001 \\
                    gamma(57) &= 0111 \ 1101 \ 1001
                \end{align*}
            \end{itemize}

            \item 800

            \textbf{Answer:}

            \begin{itemize}
                \item Converting to binary:
                \begin{tabular}{lcc}
                    \bindiv{2}{800} && Remainders\\
                    \bindiv{2}{400} &---& 0\\
                    \bindiv{2}{200} &---& 0\\
                    \bindiv{2}{100} &---& 0\\
                    \bindiv{2}{50} &---& 0\\
                    \bindiv{2}{25} &---& 0\\
                    \bindiv{2}{12} &---& 1\\
                    \bindiv{2}{6} &---& 0\\
                    \bindiv{2}{3} &---& 0\\
                    \bindiv{2}{1} &---& 1\\
                    \remainder{1} &---& 1 
                \end{tabular}

                12-bit representation: $0011 \ 0010 \ 0000$

                \item Converting to gamma coding:
                
                \begin{align*}
                    unary(floor(log_2(800))) &= unary(9) \\
                    &= 1111111110 \\
                    binary(x-2^{(floor(log_2(x)))}) &= binary(800 - 512) \\
                    &= binary(288) \\
                    &= 100100000 \\
                    gamma(800) &= 0111 \ 1111 \ 1101 \ 0010 \ 0000
                \end{align*}
            \end{itemize}

        \end{enumerate}

        \item (20\%) Below is a bit sequence for a gamma encoded gap list (as described in Chapter 5 of IIR and the lecture materials). Decode the gap list and reconstruct the corresponding list of Doc-IDs. Spaces are added for ease of reading.
        Hint: there are six Doc-IDs.
        \[ 1110 \ 0101 \ 1110 \ 1010 \ 1001 \ 1111 \ 1000 \ 0110 \ 1101 \ 1101 \]

        \textbf{Answer:} Regrouping the bit sequence using the following algorithm:
        \begin{itemize}
            \item Read the 1 bits until the first zero and store it as $magnitude$ of the unary code
            \item Read the $magnitude$ number of bits as $b=\text{binary code}$
            \item Compute $u=2^{magnitude}$
            \item Add $u$ to the decimal value of the binary code, $u + dec(b)$
        \end{itemize}

        Therefore,
        \begin{align*}
            &\{1110 010 \ 11110 1010 \ 10 0 \ 1111110 000110 \ 110 11 \ 10 1 \}\\
            &= \{2^3 + dec(010), \ 2^4 +  dec(1010), \ 2^1 + dec(0), \ 2^6 + dec(000110), \ 2^2 + dec(11), \ 2^1 + dec(1)\} \\
            &= \{2^3 + 2, \ 2^4 +  10, \ 2^1 + 0, \ 2^6 + 6, \ 2^2 + 3, \ 2^1 + 1\} \\
            &= \{8 + 2, \ 16 + 10, \ 2 + 0, \ 64 + 6, \ 4 + 3, \ 2 + 1\} \\
            &= \{10, \ 26, \ 2, \ 70, \ 7, \ 3\} \\
        \end{align*}

        \item (30\%) Researchers have proposed other schemes for gap list compression in inverted files. Some of these are slightly less space efficient, but may have faster implementations on modern hardware. Consider the following scheme called Simple-9 (Anh \& Moffat, "Inverted Index Compression Using Word-Aligned Binary Codes", 2005).
        
        In Simple-9 a 32-bit word is used to store between 1 and 28 numbers (gaps). The first four bits are used as a control to choose between nine options shown in the table below. The remaining 28 bits of each 32-bit word store numbers (gaps) using the same number of bits. For example, if the next 7 gaps are each between 1 and 16, then each of the 7 gaps will be represented in 4 bits. Note that since there are no gaps of size zero in a postings list, a gap of 1 is represented in 4 bits as "0000" and a gap of 16 is represented as "1111" -- in other words for Simple-9 we store a gap, $g$, as the integer $g-1$ in binary. The following table shows the control bits, the number of integers represented, the length of each code in bits, and the number of unused bits wasted at the end. In this problem, any leftover bits are set to 0.

        \begin{center}
            \begin{tabular}{| c | >{\centering\arraybackslash}p{3cm} | >{\centering\arraybackslash}p{3cm} | c | c |}
            \hline
            \textbf{4-bit control} & \textbf{\# codes in 28-bit block} & \textbf{code length in bits} & \textbf{leftover bits} & \textbf{possible gaps} \\
            \hline
            0000 & 28 & 1 & 0 & 1 or 2 \\
            0001 & 14 & 2 & 0 & 1 to 4 \\
            0010 & 9 & 3 & 1 & 1 to 8 \\
            0011 & 7 & 4 & 0 & 1 to 16 \\
            0100 & 5 & 5 & 3 & 1 to 32 \\
            0101 & 4 & 7 & 0 & 1 to 128 \\
            0110 & 3 & 9 & 1 & 1 to 512 \\
            0111 & 2 & 14 & 0 & 1 to 16384 \\
            1000 & 1 & 28 & 0 & 1 to $2^{28}$ \\
            \hline
            \end{tabular}
        \end{center}

        Using Simple-9 encode a posting list for the following nine Doc-IDs. Please write the bits four at a time, with a space between each four. Recall that you will first have to create a gap list before encoding gaps. Work carefully and do show your work. Doc-IDs: \{4, 12, 15, 35, 36, 52, 102, 118, 218\}

        \textbf{Answer:} Gap list of \{4, 12, 15, 35, 36, 52, 102, 118, 218\}:
        
        \{4, 8, 3, 20, 1, 16, 50, 16, 100\}

        Mapping the gaps to their range of possible gaps
        \begin{center}
            \begin{tabular}{| c | c | c |}
            \hline
            \textbf{gap} & \textbf{range} & \textbf{4-bit control} \\
            \hline
            4 & 1 to 4 & 0001 \\
            8 & 1 to 8 & 0010 \\
            3 & 1 to 4 & 0001 \\
            20 & 1 to 32 & 0100 \\
            1 & 1 or 2 & 0000 \\
            16 & 1 to 16 & 0011 \\
            50 & 1 to 128 & 0101 \\
            16 & 1 to 16 & 0011 \\
            100 & 1 to 128 & 0101 \\
            \hline
            \end{tabular}
        \end{center}

        Filling in the first 5 gaps, \{4, 8, 3, 20, 1\}, into the 0100 selector, which accepts gaps ranging 1 to 32:
        \begin{center}
            \begin{tabular}{| c | >{\centering\arraybackslash}p{3cm} | >{\centering\arraybackslash}p{3cm} | c | >{\centering\arraybackslash}p{3cm} |}
            \hline
            \textbf{4-bit control} & \textbf{\# codes in 28-bit block} & \textbf{code length in bits} & \textbf{gaps $g$} & \textbf{binary $g-1$}\\
            \hline
            0100 & 5 & 5 & \{4, 8, 3, 20, 1\} & \{00011, 00111, 00010, 10011, 00000\} \\
            \hline
            \end{tabular}
        \end{center}

        Filling in the latter 4 gaps, \{16, 50, 16, 100\}, into the 0101 selector, which accepts gaps ranging 1 to 128:
        \begin{center}
            \begin{tabular}{| c | >{\centering\arraybackslash}p{3cm} | >{\centering\arraybackslash}p{3cm} | c | >{\centering\arraybackslash}p{3cm} |}
            \hline
            \textbf{4-bit control} & \textbf{\# codes in 28-bit block} & \textbf{code length in bits} & \textbf{gaps $g$} & \textbf{binary $g-1$}\\
            \hline
            0101 & 4 & 7 & \{16, 50, 16, 100\} & \{0001111, 0110001, 0001111, 1100011\} \\
            \hline
            \end{tabular}
        \end{center}

        Combining the tables:
        \begin{center}
            \begin{tabular}{| c | c | >{\centering\arraybackslash}p{3cm} | >{\centering\arraybackslash}p{7cm} |}
            \hline
            \textbf{element} & \textbf{4-bit control} & \textbf{gaps (decimal)} & \textbf{gaps (binary with leftover bits)} \\
            \hline
            0 & 0100 & \{4, 8, 3, 20, 1\} & 0001 \ 1001 \ 1100 \ 0101 \ 0011 \ 0000 \ 0000 \\
            1 & 0101 & \{16, 50, 16, 100\} & 0001 \ 1110 \ 1100 \ 0100 \ 0111 \ 1110 \ 0011 \\
            \hline
            \end{tabular}
        \end{center}

    \end{enumerate}

\end{document}
