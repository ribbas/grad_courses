\documentclass[11pt]{article}

\usepackage{amsmath}
\usepackage{amssymb}
\usepackage{array}
\usepackage{caption}
\usepackage{graphicx}

\usepackage[autostyle, english=american]{csquotes}
\MakeOuterQuote{"}
\captionsetup[table]{skip=1pt}

\newcolumntype{M}[1]{>{\centering\arraybackslash}m{#1}}

% margins
\topmargin=-0.45in
\evensidemargin=0in
\oddsidemargin=0in
\textwidth=6.5in
\textheight=9.0in
\headsep=0.25in

\title{605.744: Information Retrieval \\ Problem Set (Module 9)}
\author{Sabbir Ahmed}
\date{\today}

\begin{document}
\maketitle

    \begin{enumerate}

        \item (30\%) Give a short definition or explanation of the following concepts:
        \begin{itemize}
            \item web spam

            \textbf{Answer:} Content on the web that is designed to be favorable in relevance even though they may be completely irrelevant.

            \item Broders' taxonomy

            \textbf{Answer:} Classification of search queries by users into 3 categories: informational, navigational, and transactional.

            \item out-degree

            \textbf{Answer:} In a directed graph, out-degree is the number of edges going out of a vertex.

            \item robots exclusion protocol

            \textbf{Answer:} Also known as robots.txt, it's used by web pages to inform crawlers on which portions to avoid indexing.

            \item priority queue (in the context of web crawling)

            \textbf{Answer:} 

        \end{itemize}

        \item (20\%) Describe in your own words the process described in the course text to efficiently identify near duplicate documents in a large collection.

        \textbf{Answer:}

        \item For this problem work with the directed web graph shown below. In the graph there are six nodes: Y, B, F, G, T, R (for the websites Yahoo, Bing, Facebook, Google, Twitter, and Reddit). Use a teleport probability of 0.20. Assume no other pages or links exist beside those shown in the figure.

        \begin{figure}[!ht]
            \includegraphics[scale=0.5]{graph.png}
            \centering
        \end{figure}

        \begin{enumerate}
            \item (15\%) Provide (i.e., write) the six recurrence equations that indicate how to iteratively calculate the PageRank score of each page at time t given scores from time t-1.

            \textbf{Answer:}
            Using the recurrence equation:
            \begin{equation}
                PR(a)=\frac{q}{N}+(1-q)\sum_{i=1}^{n}\frac{PR(p_i)}{C(p_i)}
            \end{equation}
            With the values of the sites being initialized to equal probabilities:
            \begin{table}[!ht]
                \centering
                \begin{tabular}[t]{|c|c|c|c|c|c|c|}
                    \hline
                    \textbf{Time} & \textbf{Y} & \textbf{B} & \textbf{F} & \textbf{G} & \textbf{T} & \textbf{R} \\
                    \hline
                    t = 0 & 0.167 & 0.167 & 0.167 & 0.167 & 0.167 & 0.167
                    \\ \hline
                \end{tabular}
            \end{table}
            \begin{align*}
                PR(Y,t_{i}) &= \frac{0.20}{6}+0 \\
                PR(B,t_{i}) &= \frac{0.20}{6}+(0.80)\left(\frac{PR(Y,t_{i-1})}{C(Y)}+\frac{PR(F,t_{i-1})}{C(F)}+\frac{PR(G,t_{i-1})}{C(G)}\right) \\
                PR(F,t_{i}) &= \frac{0.20}{6}+(0.80)\left(\frac{PR(R,t_{i-1})}{C(R)}+\frac{PR(T,t_{i-1})}{C(T)}\right) \\
                PR(G,t_{i}) &= \frac{0.20}{6}+(0.80)\left(\frac{PR(B,t_{i-1})}{C(B)}+\frac{PR(Y,t_{i-1})}{C(Y)}\right) \\
                PR(T,t_{i}) &= \frac{0.20}{6}+(0.80)\left(\frac{PR(B,t_{i-1})}{C(B)}\right) \\
                PR(R,t_{i}) &= \frac{0.20}{6}+(0.80)\left(\frac{PR(T,t_{i-1})}{C(T)}\right)
            \end{align*}

            \item (25\%) Using the brute-force iterative method of calculation shown in the video lecture calculate two iterations of PageRank scores for each page in the graph. Be sure to show scores at times t=0, t=1, and finally at t=2. Report scores using three digits of precision (e.g., 0.247, not 0.2 or 0.24696485932). Show work and do not merely provide a table of values.

            \textbf{Answer:}
            \begin{align*}
                PR(Y,t_{1}) &= \frac{0.20}{6}+0 \\
                &= \frac{1}{30} \\
                &= 0.033
            \end{align*}
            \begin{align*}
                PR(B,t_{1}) &= \frac{0.20}{6}+(0.80)\left(\frac{PR(Y,t_{0})}{C(Y)}+\frac{PR(F,t_{0})}{C(F)}+\frac{PR(G,t_{0})}{C(G)}\right) \\
                &= \frac{1}{30}+(0.80)\frac{1}{6}\left(\frac{1}{2}+\frac{1}{1}+\frac{1}{1}\right) \\
                &= \frac{1}{30}+\frac{1}{3} \\
                &= 0.367 \\
                PR(B,t_{2}) &= \frac{1}{30}+(0.80)\left(\frac{0.033}{2}+\frac{0.233}{1}+\frac{0.167}{1}\right) \\
                &= 0.367
            \end{align*}
            \begin{align*}
                PR(F,t_{1}) &= \frac{0.20}{6}+(0.80)\left(\frac{PR(R,t_{0})}{C(R)}+\frac{PR(T,t_{0})}{C(T)}\right) \\
                &= \frac{1}{30}+(0.80)\frac{1}{6}\left(\frac{1}{1}+\frac{1}{2}\right) \\
                &= \frac{1}{30}+\frac{1}{5} \\
                &= 0.233 \\
                PR(F,t_{2}) &= \frac{1}{30}+(0.80)\left(\frac{0.100}{1}+\frac{0.100}{2}\right) \\
                &= 0.153
            \end{align*}
            \begin{align*}
                PR(G,t_{1}) &= \frac{0.20}{6}+(0.80)\left(\frac{PR(B,t_{0})}{C(B)}+\frac{PR(Y,t_{0})}{C(Y)}\right) \\
                &= \frac{1}{30}+(0.80)\frac{1}{6}\left(\frac{1}{2}+\frac{1}{2}\right) \\
                &= 0.167 \\
                PR(G,t_{2}) &= \frac{1}{30}+(0.80)\left(\frac{0.367}{2}+\frac{0.033}{2}\right) \\
                &= 0.193
            \end{align*}
            \begin{align*}
                PR(T,t_{1}) &= \frac{0.20}{6}+(0.80)\left(\frac{PR(B,t_{0})}{C(B)}\right) \\
                &= \frac{1}{30}+(0.80)\frac{1}{6}\left(\frac{1}{2}\right) \\
                &= \frac{1}{30}+\frac{1}{15} \\
                &= 0.100 \\
                PR(T,t_{2}) &= \frac{1}{30}+(0.80)\left(\frac{0.367}{2}\right) \\
                &= 0.180
            \end{align*}
            \begin{align*}
                PR(R,t_{1}) &= \frac{0.20}{6}+(0.80)\left(\frac{PR(T,t_{0})}{C(T)}\right) \\
                &= \frac{1}{30}+(0.80)\frac{1}{6}\left(\frac{1}{2}\right) \\
                &= \frac{1}{30}+\frac{1}{15} \\
                &= 0.100 \\
                PR(R,t_{2}) &= \frac{1}{30}+(0.80)\left(\frac{0.100}{2}\right) \\
                &= 0.073
            \end{align*}
            \begin{table}[!ht]
                \centering
                \begin{tabular}[t]{|c|c|c|c|c|c|c|}
                    \hline
                    & \textbf{Y} & \textbf{B} & \textbf{F} & \textbf{G} & \textbf{T} & \textbf{R} \\
                    \hline
                    t = 0 & 0.167 & 0.167 & 0.167 & 0.167 & 0.167 & 0.167
                    \\ \hline
                    t = 1 & 0.033 & 0.367 & 0.233 & 0.167 & 0.100 & 0.100
                    \\ \hline
                    t = 2 & 0.033 & 0.367 & 0.153 & 0.193 & 0.180 & 0.073
                    \\ \hline
                \end{tabular}
            \end{table}

            \item (5\%) Which page (or pages) has/have the lowest PageRank score after two iterations?

            \textbf{Answer:} Yahoo has the lowest PageRank score after two iterations with a value of 0.033.

            \item (5\%) Which page (or pages) has/have the highest PageRank score after two iterations?

            \textbf{Answer:} Bing has the highest PageRank score after two iterations with a value of 0.367.

        \end{enumerate}

    \end{enumerate}

\end{document}
