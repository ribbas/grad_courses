\documentclass[11pt]{article}

\usepackage{graphicx}
\usepackage{amsmath}
\usepackage{listings}

% Margins
\topmargin=-0.45in
\evensidemargin=0in
\oddsidemargin=0in
\textwidth=6.5in
\textheight=9.0in
\headsep=0.25in

\title{605.744: Information Retrieval \\ Problem Set (Module 2)}
\author{Sabbir Ahmed}
\date{\today}

\begin{document}
\maketitle	

    \begin{enumerate}

        \item (10\%) What advantage(s) do balanced binary trees have over hashtables when used to store our dictionary of indexing terms?

        \textbf{Answer:}

        \item (30\%) Character n-gram overlap is used for both automated spelling correction and name matching (i.e., deciding whether two names might be the same, a common database problem known as "record linkage"). Using a character 3-gram representation, how many distinct n-grams do "CHEONGSONG" and "CHEONMACHONG" have in common? What is the Dice-coefficient score for these two strings using 3-grams? What is the Dice score using 4-grams instead? Which score is higher? Note: although there is nothing conceptually wrong in doing so, for this problem, do not use "padded" n-grams (e.g., '\$' or '\_' symbols marking the beginning and end of the strings).

        \textbf{Answer:}
        
        The following are the 3-grams of:
        
        "CHEONGSONG": \[ X = \{CHE, HEO, EON, ONG, NGS, GSO, SON\}, |X| = 7 \]

        "CHEONMACHONG": \[ Y = \{CHE, HEO, EON, ONM, NMA, MAC, ACH, CHO, HON, ONG\}, |Y| = 10 \]

        Both of the words: \[ X \cap Y = \{CHE, HEO, EON, ONG\}, |X \cap Y| = 4 \]

        The Dice-coefficient is computed with the formula: $2|X \cap Y| / (|X|+|Y|)$. Then, \[ 2(4) / (7 + 10) = 8 / 17 = 0.47 \]

        The following are the 4-grams of:
        
        "CHEONGSONG": \[ X = \{CHEO, HEON, EONG, ONGS, NGSO, GSON, SONG\}, |X| = 7 \]

        "CHEONMACHONG": \[ Y = \{CHEO, HEON, EONM, ONMA, NMAC, MACH, ACHO, CHON, HONG\}, |Y| = 9 \]

        Both of the words: \[ X \cap Y = \{CHEO, HEON\}, |X \cap Y| = 2 \]

        The Dice-coefficient is: $2|X \cap Y| / (|X|+|Y|)$. Then, \[ 2(2) / (7 + 9) = 4 / 16 = 0.25 \]

        The Dice-coefficient for 3-grams is higher.

        \item (30\%) Compute the edit distance (or Levenshtein distance) for these two pairs of strings: (a) "CHEBYSHEV" and "TSCHEBYSCHEF"; and (b) "LEVINSTINE" and "LEVENSHTEIN". Then report a sequence of transformations for that cost that converts one string into the other. You should use unit costs for each operation: insertion, deletion, or substitution; that is, each step has a cost of 1. Note, you do not need to write a program or produce any code for this problem - these examples can be easily determined by pencil and paper - you do not need to construct a table as the example in the textbook.

        Trivia: Computing edit distance is a classic dynamic programming problem with an $O(N^2)$ complexity. However, the decision problem "Do strings $x$ and $y$ (say each of length $N$) have an edit distance $<= k$ can be solved in $O(kN)$, which is subquadratic. See: Esko Ukkonen's paper 'Algorithms for approximate string matching' (1985).

        \textbf{Answer:}

        \begin{enumerate}
            \item "CHEBYSHEV" and "TSCHEBYSCHEF"
            \begin{enumerate}
                \item Insert T: $\text{"\_CHEBYSHEV" -> "TCHEBYSHEV"}$
                \item Insert S: $\text{"T\_CHEBYSHEV" -> "TSCHEBYSHEV"}$
                \item Insert C: $\text{"TSCHEBYS\_HEV" -> "TSCHEBYSCHEV"}$
                \item Substitute F with V: $\text{"TSCHEBYSCHE\underline{V}" -> "TSCHEBYSCHEF"}$
            \end{enumerate}
            The edit distance is 4.

            \item "LEVINSTINE" and "LEVENSHTEIN"
            \begin{enumerate}
                \item Substitute I with E: $\text{"LEV\underline{I}NSTINE" -> "LEVENSTINE"}$
                \item Insert H: $\text{"LEVENS\_TINE" -> "LEVENSHTINE"}$
                \item Insert E: $\text{"LEVENSHT\_INE" -> "LEVENSHTEINE"}$
                \item Delete E: $\text{"LEVENSHTEIN\underline{E}" -> "LEVENSHTEIN"}$
            \end{enumerate}
            The edit distance is 4.

        \end{enumerate}

        \item (30\%) Following the method described in the textbook (or lecture materials), calculate Soundex codes for the strings: (a) "Stanford" and (b) "Georgetown"? Show intermediate steps to produce the final code.

        \textbf{Answer:}
        \begin{enumerate}
            \item "Stanford"
            \begin{enumerate}
                \item Retain the first letter of the name and change all other occurrences of a, e, i, o, u, y, h, w to 0: Stanford -> St0nf0rd
                \item Replace consonants with their corresponding digits: St0nf0rd -> S3051063
                \item Remove all zeros: S3051063 -> S35163
                \item Return the first four positions: S35163 -> S351
            \end{enumerate}

            \item "Georgetown"
            \begin{enumerate}
                \item Retain the first letter of the name and change all other occurrences of a, e, i, o, u, y, h, w to 0: Georgetown -> G00rg0t00n
                \item Replace consonants with their corresponding digits: G00rg0t00n -> G006203005
                \item Remove all zeros: G006203005 -> G6235
                \item Return the first four positions: G6235 -> G623
            \end{enumerate}
        \end{enumerate}

    \end{enumerate}

\end{document}
