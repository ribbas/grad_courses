\documentclass[11pt]{article}

\usepackage{graphicx}
\usepackage{amsmath}
\usepackage{listings}

% Margins
\topmargin=-0.45in
\evensidemargin=0in
\oddsidemargin=0in
\textwidth=6.5in
\textheight=9.0in
\headsep=0.25in

\title{605.744: Information Retrieval \\ Problem Set (Module 2)}
\author{Sabbir Ahmed}
\date{\today}

\begin{document}
\maketitle	

    \begin{enumerate}

        \item (10\%) What advantage(s) do balanced binary trees have over hashtables when used to store our dictionary of indexing terms?

        \textbf{Answer:}

        \item (30\%) Character n-gram overlap is used for both automated spelling correction and name matching (i.e., deciding whether two names might be the same, a common database problem known as "record linkage"). Using a character 3-gram representation, how many distinct n-grams do "CHEONGSONG" and "CHEONMACHONG" have in common? What is the Dice-coefficient score for these two strings using 3-grams? What is the Dice score using 4-grams instead? Which score is higher? Note: although there is nothing conceptually wrong in doing so, for this problem, do not use "padded" n-grams (e.g., '\$' or '\_' symbols marking the beginning and end of the strings).

        \textbf{Answer:}

        \item (30\%) Compute the edit distance (or Levenshtein distance) for these two pairs of strings: (a) "CHEBYSHEV" and "TSCHEBYSCHEF"; and (b) "LEVINSTINE" and "LEVENSHTEIN". Then report a sequence of transformations for that cost that converts one string into the other. You should use unit costs for each operation: insertion, deletion, or substitution; that is, each step has a cost of 1. Note, you do not need to write a program or produce any code for this problem - these examples can be easily determined by pencil and paper - you do not need to construct a table as the example in the textbook.

        Trivia: Computing edit distance is a classic dynamic programming problem with an O(N2 ) complexity. However, the decision problem "Do strings x and y (say each of length N) have an edit distance <= k can be solved in O(kN), which is subquadratic. See: Esko Ukkonen's paper 'Algorithms for approximate string matching' (1985).

        \textbf{Answer:}

        \item (30\%) Following the method described in the textbook (or lecture materials), calculate Soundex codes for the strings: (a) "Stanford" and (b) "Georgetown"? Show intermediate steps to produce the final code.

        \textbf{Answer:}

    \end{enumerate}

\end{document}
