\documentclass[11pt]{article}

\usepackage{amsmath}
\usepackage{amssymb}
\usepackage{array}
\usepackage{caption}
\usepackage{graphicx}

\usepackage[autostyle, english = american]{csquotes}
\MakeOuterQuote{"}
\captionsetup[table]{skip=1pt}

\newcolumntype{M}[1]{>{\centering\arraybackslash}m{#1}}

% margins
\topmargin=-0.45in
\evensidemargin=0in
\oddsidemargin=0in
\textwidth=6.5in
\textheight=9.0in
\headsep=0.25in

\title{605.744: Information Retrieval \\ Problem Set (Module 8)}
\author{Sabbir Ahmed}
\date{\today}

\begin{document}
\maketitle

    \begin{enumerate}

        \item (20\%) Name three significant issues that arise when using dictionaries to translate queries in cross-language
        information retrieval and briefly explain why they create a problem.

        \textbf{Answer:}

        \item (20\%) Give two advantages and one disadvantage of using character n-gram tokenization for multilingual
        text retrieval.

        \textbf{Answer:}

        \item (20\%) For this question consider an English alphabet to consist of just 26 (lower-cased) letters, 10 digits,
        and a space character. And consider there to be exactly 10,000 characters in Chinese. Note, spaces are
        not used in written Chinese.

        \begin{enumerate}
            \item How many possible character 4-grams are there in English? Using Table 5.1 (IIR) how does this number
            compare to a typical vocabulary size when words are used?

            \textbf{Answer:}

            \item How many possible indexing terms will there be if 2-gram indexing is used for Chinese? What if 3-grams
            are used?

            \textbf{Answer:}

            \item What difficulties might occur when indexing a document collection if the vocabulary size (i.e., number of
            indexing terms) is extremely large?

            \textbf{Answer:}

        \end{enumerate}

        \item (20\%) What advantages does query translation have over document translation in cross-language
        information retrieval (CLIR)?

        \textbf{Answer:}

        \item (20\%) Briefly describe what pre-translation query expansion (sometimes called pre-translation feedback)
        is and then explain why it is helpful in dictionary-based cross-language information retrieval.

        \textbf{Answer:}

    \end{enumerate}

\end{document}
