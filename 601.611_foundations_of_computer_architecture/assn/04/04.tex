\documentclass[12pt]{article}
\usepackage[margin=1.0in]{geometry}
\usepackage{setspace}
\usepackage{titling}

\usepackage{amsmath,amsthm,amssymb}
\usepackage{graphicx}
\usepackage{enumerate}
\usepackage[shortlabels]{enumitem}

\setstretch{1.5}
\setlength{\droptitle}{-8em}

\newcommand{\specialcell}[2][c]{%
  \begin{tabular}[#1]{@{}c@{}}#2\end{tabular}}

\begin{document}

  \title{605.611 - Foundations of Computer Architecture \\ Assignment 04\vspace{-0.5em}}
  \author{Sabbir Ahmed}
  \maketitle
  \vspace{-1em}

  \begin{enumerate}
    \item Implement the table to show the steps in the multiplication using Figure 3.7.  Use 8 bit numbers.

    \begin{enumerate}
      \item 3 * 5
      \begin{center}
      \begin{tabular}{|c|c|c|c|} 
        \hline
        \textbf{Iteration} & \textbf{Multiplicand} & \textbf{Product} & \textbf{Steps} \\ 
        \hline
        0 & 0000 0011 & 0000 0000 0000 0101 & Start \\
        \hline
        1 & 0000 0011 & 0000 0001 1000 0010 & \specialcell{LSB = 1 \\ Add \\ Shift right} \\
        \hline
        2 & 0000 0011 & 0000 0000 1100 0001 & \specialcell{LSB = 0 \\ Shift right} \\
        \hline
        3 & 0000 0011 & 0000 0001 1110 0000 & \specialcell{LSB = 1 \\ Add \\ Shift right} \\
        \hline
        4 & 0000 0011 & 0000 0000 1111 0000 & \specialcell{LSB = 0 \\ Shift right} \\
        \hline
        5 & 0000 0011 & 0000 0000 0111 1000 & \specialcell{LSB = 0 \\ Shift right} \\
        \hline
        6 & 0000 0011 & 0000 0000 0011 1100 & \specialcell{LSB = 0 \\ Shift right} \\
        \hline
        7 & 0000 0011 & 0000 0000 0001 1110 & \specialcell{LSB = 0 \\ Shift right} \\
        \hline
        8 & 0000 0011 & 0000 0000 0000 1111 & \specialcell{LSB = 0 \\ Shift right} \\
        \hline
      \end{tabular}
      \end{center}

      \item 4 * 6
      \begin{center}
      \begin{tabular}{|c|c|c|c|} 
        \hline
        \textbf{Iteration} & \textbf{Multiplicand} & \textbf{Product} & \textbf{Steps} \\ 
        \hline
        0 & 0000 0100 & 0000 0000 0000 0110 & Start \\
        \hline
        1 & 0000 0100 & 0000 0000 0000 0011 & \specialcell{LSB = 0 \\ Shift right} \\
        \hline
        2 & 0000 0100 & 0000 0010 0000 0001 & \specialcell{LSB = 1 \\ Add \\ Shift right} \\
        \hline
        3 & 0000 0100 & 0000 0011 0000 0000 & \specialcell{LSB = 1 \\ Add \\ Shift right} \\
        \hline
        4 & 0000 0100 & 0000 0001 1000 0000 & \specialcell{LSB = 0 \\ Shift right} \\
        \hline
        5 & 0000 0100 & 0000 0000 1100 0000 & \specialcell{LSB = 0 \\ Shift right} \\
        \hline
        6 & 0000 0100 & 0000 0000 0110 0000 & \specialcell{LSB = 0 \\ Shift right} \\
        \hline
        7 & 0000 0100 & 0000 0000 0011 0000 & \specialcell{LSB = 0 \\ Shift right} \\
        \hline
        8 & 0000 0100 & 0000 0000 0001 1000 & \specialcell{LSB = 0 \\ Shift right} \\
        \hline
      \end{tabular}
      \end{center}
    \end{enumerate}

    \item Implement the table to show the steps in division using Figure 3.13

    \begin{enumerate}
      \item 15 / 6
      \begin{center}
      \begin{tabular}{|c|c|c|c|c|} 
        \hline
        \textbf{Iteration} & \textbf{Divisor} & \textbf{Complement} & \textbf{Remainder} & \textbf{Steps} \\ 
        \hline
        0 & 0000 0110 & 1111 1010 & 0000 0000 0000 1111 & Start \\
        \hline

        1a & 0000 0110 & 1111 1010 & 0000 0000 0001 1110 & Shift left \\
        \hline
        1b & 0000 0110 & 1111 1010 & 1111 1010 0001 1110 & Add complement \\
        \hline
        1c & 0000 0110 & 1111 1010 & 0000 0000 0001 1110 & Add divisor \\
        \hline

        2a & 0000 0110 & 1111 1010 & 0000 0000 0011 1100 & Shift left \\
        \hline
        2b & 0000 0110 & 1111 1010 & 1111 1010 0011 1100 & Add complement \\
        \hline
        2c & 0000 0110 & 1111 1010 & 0000 0000 0011 1100 & Add divisor \\
        \hline

        3a & 0000 0110 & 1111 1010 & 0000 0000 0111 1000 & Shift left \\
        \hline
        3b & 0000 0110 & 1111 1010 & 1111 1010 0111 1000 & Add complement \\
        \hline
        3c & 0000 0110 & 1111 1010 & 0000 0000 0111 1000 & Add divisor \\
        \hline

        4a & 0000 0110 & 1111 1010 & 0000 0000 1111 0000 & Shift left \\
        \hline
        4b & 0000 0110 & 1111 1010 & 1111 1010 1111 0000 & Add complement \\
        \hline
        4c & 0000 0110 & 1111 1010 & 0000 0000 1111 0000 & Add divisor \\
        \hline

        5a & 0000 0110 & 1111 1010 & 0000 0001 1110 0000 & Shift left \\
        \hline
        5b & 0000 0110 & 1111 1010 & 1111 1011 1110 0000 & Add complement \\
        \hline
        5c & 0000 0110 & 1111 1010 & 0000 0001 1110 0000 & Add divisor \\
        \hline

        6a & 0000 0110 & 1111 1010 & 0000 0011 1100 0000 & Shift left \\
        \hline
        6b & 0000 0110 & 1111 1010 & 1111 1101 1100 0000 & Add complement \\
        \hline
        6c & 0000 0110 & 1111 1010 & 0000 0011 1100 0000 & Add divisor \\
        \hline

        7a & 0000 0110 & 1111 1010 & 0000 0111 1000 0000 & Shift left \\
        \hline
        7b & 0000 0110 & 1111 1010 & 0000 0001 1000 0000 & Add complement \\
        \hline
        7c & 0000 0110 & 1111 1010 & 0000 0001 1000 0001 & Invert LSB \\
        \hline

        8a & 0000 0110 & 1111 1010 & 0000 0011 0000 0010 & Shift left \\
        \hline
        8b & 0000 0110 & 1111 1010 & 1111 1101 0000 0010 & Add complement \\
        \hline
        8c & 0000 0110 & 1111 1010 & 0000 0011 0000 0010 & Add divisor \\
        \hline
      \end{tabular}
      \end{center}

      \item 13 / 3
      \begin{center}
      \begin{tabular}{|c|c|c|c|c|} 
        \hline
        \textbf{Iteration} & \textbf{Divisor} & \textbf{Complement} & \textbf{Remainder} & \textbf{Steps} \\ 
        \hline
        0 & 0000 0011 & 1111 1101 & 0000 0000 0000 1101 & Start \\
        \hline

        1a & 0000 0011 & 1111 1101 & 0000 0000 0001 1010 & Shift left \\
        \hline
        1b & 0000 0011 & 1111 1101 & 1111 1101 0001 1010 & Add complement \\
        \hline
        1c & 0000 0011 & 1111 1101 & 0000 0000 0001 1010 & Add divisor \\
        \hline

        2a & 0000 0011 & 1111 1101 & 0000 0000 0011 0100 & Shift left \\
        \hline
        2b & 0000 0011 & 1111 1101 & 1111 1101 0011 0100 & Add complement \\
        \hline
        2c & 0000 0011 & 1111 1101 & 0000 0000 0011 0100 & Add divisor \\
        \hline

        3a & 0000 0011 & 1111 1101 & 0000 0000 0110 1000 & Shift left \\
        \hline
        3b & 0000 0011 & 1111 1101 & 1111 1101 0110 1000 & Add complement \\
        \hline
        3c & 0000 0011 & 1111 1101 & 0000 0000 0110 1000 & Add divisor \\
        \hline

        4a & 0000 0011 & 1111 1101 & 0000 0000 1101 0000 & Shift left \\
        \hline
        4b & 0000 0011 & 1111 1101 & 1111 1101 1101 0000 & Add complement \\
        \hline
        4c & 0000 0011 & 1111 1101 & 0000 0000 1101 0000 & Add divisor \\
        \hline

        5a & 0000 0011 & 1111 1101 & 0000 0001 1010 0000 & Shift left \\
        \hline
        5b & 0000 0011 & 1111 1101 & 1111 1110 1010 0000 & Add complement \\
        \hline
        5c & 0000 0011 & 1111 1101 & 0000 0001 1010 0000 & Add divisor \\
        \hline

        6a & 0000 0011 & 1111 1101 & 0000 0011 0100 0000 & Shift left \\
        \hline
        6b & 0000 0011 & 1111 1101 & 0000 0000 0100 0000 & Add complement \\
        \hline
        6c & 0000 0011 & 1111 1101 & 0000 0000 0100 0001 & Invert LSB \\
        \hline

        7a & 0000 0011 & 1111 1101 & 0000 0000 1000 0010 & Shift left \\
        \hline
        7b & 0000 0110 & 1111 1010 & 1111 1101 1000 0010 & Add complement \\
        \hline
        7c & 0000 0110 & 1111 1010 & 0000 0000 1000 0010 & Add divisor \\
        \hline

        8a & 0000 0110 & 1111 1010 & 0000 0001 0000 0100 & Shift left \\
        \hline
        8b & 0000 0110 & 1111 1010 & 1111 1101 0000 0100 & Add complement \\
        \hline
        8c & 0000 0110 & 1111 1010 & 0000 0001 0000 0100 & Add divisor \\
        \hline
      \end{tabular}
      \end{center}

    \end{enumerate}

    \item Implement the hardware multiplication and division algorithms in software.  You can use addition and subtraction and any bit-wise operations you need.  You can do this in any language you choose.

  \end{enumerate}

\end{document}
