\documentclass[12pt]{article}
\usepackage[margin=1.0in]{geometry}
\usepackage{setspace}
\usepackage{titling}

\usepackage{amsmath,amsthm,amssymb}
\usepackage{graphicx}
\usepackage{enumerate}
\usepackage[shortlabels]{enumitem}

\newcommand{\answeritem}{\global\answertrue\item}
\newcommand{\perhapsanswer}{%
  \ifanswer
    $\blacksquare$ \global\answerfalse
  \else
    $\square$ \global\answerfalse
  \fi
}
\newif\ifanswer

\setstretch{1.5}
\setlength{\droptitle}{-8em}

\usepackage[thinlines]{easytable}

\begin{document}

  \title{605.611 - Foundations of Computer Architecture \\ Assignment 02\vspace{-0.5em}}
  \author{Sabbir Ahmed}
  \maketitle
  \vspace{-1em}

  \begin{enumerate}

    % 1
    \item Convert the following fixed point numbers to binary fixed point. Give both the actual values, and normalize the values so that they have a binary 1 as the value for the left of the decimal point.
    \begin{enumerate}
      \item 7.25
      \item 13.5
      \item 0.5625
      \item 0.125
      \item 127.625
      \item 51,025.025
      \item 7.1
      \item 5.2
    \end{enumerate}

    % 5
    \setcounter{enumi}{4}
    \item Convert the following from decimal to excess 127 format. Write your answers as hexadecimal digits.
    \begin{enumerate}
      \item -4

      \item 4

      \setcounter{enumii}{3}
      \item 7

      \item -7

    \end{enumerate}

    % 8
    \setcounter{enumi}{7}
    \item Single precision floating point numbers have 7 digit decimal precision and double floating point numbers have 15 digit precision. Explain how these precision values are arrived at, and what they mean.

    % 9
    \item Convert the following numbers to IEEE 754 single precision numbers. Give your answers as hexadecimal numbers (do not give me binary, I cannot read it accurately. I WILL misread it and you WILL lose points).
    \begin{enumerate}
      \item 7.25
      \item 13.5
      \item 0.5625
      \item 0.125
      \item 127.625
      \item 51025.025
    \end{enumerate}

    % 10
    \item For each of the following truth tables:
    \begin{enumerate}
      \item Give the DNF equation for the table.
      \item Give the minimal equation.
      \item Using Boolean algebra show the two Boolean equations are equivalent.
      \item Draw the circuit in Logisim. Be prepared to draw the circuit by hand.
    \end{enumerate}

    \begin{TAB}(e,1cm,1cm){|c|c|c|c|}{|c|cccccccc|}
      \textbf{A} & \textbf{B} & \textbf{C} & \textbf{f(A,B,C)} \\
      0 & 0 & 0 & 0 \\
      0 & 0 & 1 & 1 \\
      0 & 1 & 0 & 1 \\
      0 & 1 & 1 & 0 \\
      1 & 0 & 0 & 0 \\
      1 & 0 & 1 & 1 \\
      1 & 1 & 0 & 1 \\
      1 & 1 & 1 & 0
    \end{TAB}

    \begin{TAB}(e,1cm,1cm){|c|c|c|c|}{|c|cccccccc|}
      \textbf{A} & \textbf{B} & \textbf{C} & \textbf{f(A,B,C)} \\
      0 & 0 & 0 & 1 \\
      0 & 0 & 1 & 1 \\
      0 & 1 & 0 & 1 \\
      0 & 1 & 1 & 1 \\
      1 & 0 & 0 & 0 \\
      1 & 0 & 1 & 1 \\
      1 & 1 & 0 & 0 \\
      1 & 1 & 1 & 1
    \end{TAB}

    \begin{TAB}(e,1cm,1cm){|c|c|c|c|}{|c|cccccccc|}
      \textbf{A} & \textbf{B} & \textbf{C} & \textbf{f(A,B,C)} \\
      0 & 0 & 0 & 1 \\
      0 & 0 & 1 & 0 \\
      0 & 1 & 0 & 0 \\
      0 & 1 & 1 & 1 \\
      1 & 0 & 0 & 1 \\
      1 & 0 & 1 & 1 \\
      1 & 1 & 0 & 1 \\
      1 & 1 & 1 & 1
    \end{TAB}

  \end{enumerate}

\end{document}
