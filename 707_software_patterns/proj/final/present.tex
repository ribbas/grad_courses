\section{Software Patterns Present in SST}
The following patterns can be observed to have been already implemented in the project:
\begin{enumerate}
    \item Abstract factory pattern
    \item Factory method pattern
    \item Prototype pattern
    \item Singleton pattern
    \item Strategy pattern
\end{enumerate}
Other patterns are present in the project, such as C++ idioms (Include Guard Macro, enable\_if, etc.)

\subsection{Abstract Factory/Factory Method}
The abstract factory and factory method patterns are present in the \texttt{SST::Factory} class. In the repository, the class can be located at \texttt{factory.h}. In the repository, it is used to create several concrete classes, including \texttt{Component} and \texttt{Module} objects. The class also provides templated variadic methods to create concrete classes of generic classes, such as
\begin{lstlisting}[style=customC++]
// src/sst/core/factory.h
/*
 * General function to create a given base class.
 *
 * @param type
 * @param params
 * @param args Constructor arguments
 */
template<class Base, class ... CtorArgs>
Base* Create(const std::string& type, CtorArgs&& ... args)
\end{lstlisting}

\subsection{Singleton}
The singleton pattern is present in the \texttt{SST::Factory} class. In the repository, the class can be located at \texttt{factory.h}. The class is used to instantiate other concrete simulation classes. SST requires simulation objects to be synchronized throughout the kernel, especially since they can be running on a distributed system where race conditions can become major issues. The software forces these simulation objects to be Singletons.

\subsection{Strategy pattern}
The Strategy pattern is present in the \texttt{SST::Core::Serialization::serializer} class. The class is implemented throughout multiple files in \texttt{serialization}, where it is overloaded in the files with various parameter types, with all the various versions of the class simply overloading the function call operator (\texttt{operator()}).

\subsection{Prototype pattern}
The Prototype pattern is present in the project, although in a very limited manner. Select \texttt{SST::Core::Interfaces} classes implement a \texttt{clone} method to provide the ability to copy instances of themselves. The \\ \texttt{SST::Core::Interfaces::StringEvent} class provides a shallow copy of itself to instantiate its ConcretePrototype, while \texttt{SST::Core::Interfaces::SimpleNetwork} performs a deep copy.
