\documentclass[titlepage]{article}
\usepackage[margin=1.0in]{geometry}
\usepackage{hyperref}

\title{Final Project Draft}
\author{Sabbir Ahmed}
\date{\today}

\begin{document}

\maketitle

\section{Introduction}
In software engineering, design patterns are general, reusable solutions to commonly occuring problems \cite{source-making}. It is generally considered good practice to integrate design patterns into software products, especially large projects, since it allows the developers to focus their time and attention towards specific implementations. The purpose of this draft is to introduce an open-source project and present analysis on the software patterns within the implementation.

\section{Structural Simulation Toolkit (SST)}
The software that is being focused on in the final project is Structural Simulation Toolkit (SST). It is a simulation framework that prioritizes high performance computing (HPC) models. Since SST is a large scale project with many stable extensions implemented for its kernel, the scope of the project will be limited to specific sections of the core repository. The repository is hosted on GitHub: https://github.com/sstsimulator/sst-core.

\section{Software Patterns Present in SST}
The following patterns can be observed to have been already implemented in the project:
\begin{enumerate}
    \item Factory method pattern
    \item Singleton pattern
    \item Template method pattern
    \item Builder pattern
\end{enumerate}

Other patterns are present in the project, such as 

\subsection{Factory method pattern}

\subsection{Singleton pattern}

\subsection{Template method pattern}
This pattern is implemented throughout the project through C++ function templates.

\subsection{Builder pattern}

\section{Suggested Software Patterns in SST}

\subsection{Strategy pattern}

\subsection{Decorator pattern}

\bibliographystyle{plain}
\bibliography{ref}

% \begin{thebibliography}{1}
%     \bibitem{patterndef} https://sourcemaking.com/design\_patterns
% \end{thebibliography}


\end{document}
